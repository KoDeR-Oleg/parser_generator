\chapter{Интерфейсы}


\Section{Tree}

Интерфейс деревьев, для парсинга которых генерируется парсер. Задаёт три абстрактных метода:
\\

\begin{lstlisting}
def get_value(self, tree_path: TreePath) -> str
\end{lstlisting}
Метод возвращает значение элемента дерева, на который указывает $tree\_path$.
\\

\begin{lstlisting}
def get_elements(self, tree_path: TreePath) -> List[Tree]
\end{lstlisting}
Метод возвращает список поддеревьев, которые удовлетворяют пути $tree\_path$.
\\

\begin{lstlisting}
def get_iter(self)
\end{lstlisting}
Переопределение базового метода. Возвращает итератор по всем элементам дерева.

\Section{TreePath}

Интерфейс путей в дереве. Задаёт основные операции, которые можно производить с путями:
\\

\begin{lstlisting}
def get_relative_path(self, tree_path) -> TreePath
\end{lstlisting}

Метод возвращает путь относительно пути $tree\_path$. Необходима поддержка лишь случая, когда $tree\_path$ является префиксом текущего пути. Без обобщения на пути вида <<../../>>.
\\

\begin{lstlisting}
def get_common_prefix(self, tree_path, in_block=False) -> TreePath
\end{lstlisting}

Метод возвращает наибольший общий префикс путей. Если флак $in\_block$ выставлен, то для путей производится их обощение по индексам. Иначе определяется точное совпадение.
\\

\begin{lstlisting}
def len(self) -> int
\end{lstlisting}

Метод возвращает длину пути -- целое число.
\\

\begin{lstlisting}
def drop_for_len(self, len) -> TreePath
\end{lstlisting}

Обрезание пути до заданной длины.
\\

\begin{lstlisting}
def concat(self, tree_path) -> TreePath
\end{lstlisting}

Конкатенация путей. Входной параметр должен содержать относительный путь.
\\

\Section{Markup}

Интерфейс разметки. Необходимо определить три поля и один метод:
\\

\begin{lstlisting}
file
\end{lstlisting}

Относительный путь до файла, которому соответствует разметка.
\\

\begin{lstlisting}
type
\end{lstlisting}

Тип разметки, поддерживаемый классом $MarkupTypeRegistry$.
\\

\begin{lstlisting}
components
\end{lstlisting}

Список элементов разметки, определённых в заданном файле.
\\

\begin{lstlisting}
def add(self, component)
\end{lstlisting}

Метод добавления новой компоненты в список $components$.
\\


\Section{Algorithm}

Интерфейс основного алгоритма. Задаёт два метода, которые определяют схему взаимодействия с ним:
\\

\begin{lstlisting}
def learn(self, markup_list: List[Markup])
\end{lstlisting}

Метод обучения парсера на списке разметок. На вход принимает список разметок одного типа.
\\

\begin{lstlisting}
def parse(self, raw_page: str) -> ParserResult:
\end{lstlisting}

Метод для парсинга страницы. Следует вызывать после вызова обучения парсера. Возвращает элемент типа $ParserResult$.
