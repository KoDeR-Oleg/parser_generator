\documentclass[a4paper,14pt]{extbook}
\usepackage[utf8]{inputenc}
\usepackage[T2A]{fontenc}
\usepackage[english,russian]{babel}
\usepackage[dvips]{graphicx}
\usepackage{amsmath}
\usepackage{amssymb}
\usepackage{amsfonts}
\usepackage{makeidx}
\makeindex
\usepackage{longtable}
\usepackage{fontspec}
\usepackage{courier}

\usepackage[indentfirst,compact,topmarks,calcwidth,pagestyles]{titlesec}
\usepackage{titletoc}

\usepackage{listings}

\usepackage{pst-tree}

\linespread{1.3}
\setlength{\parskip}{2mm plus 1mm minus 1mm} 

\usepackage[left=3cm,right=2cm,top=2.5cm,bottom=1.5cm,bindingoffset=0cm]{geometry}


\newcommand{\Subsection}[1]{
	\subsubsection*{#1}\addcontentsline{toc}{subsection}{#1}
}

\newcommand{\Section}[1]{
	\subsection*{#1}\addcontentsline{toc}{section}{#1}
}

\makeatletter
    \renewcommand{\l@chapter}[2]{\vspace{0.2cm}\@tempdima=1.0cm
        \hbox to\textwidth{{\bf \noindent Глава #1} \hfil}}
    \renewcommand{\thechapter}{\arabic{chapter}.}
    \renewcommand{\@makechapterhead}[1]{
       {\vspace*{1.0cm}
        \noindent
        \Large \bf \@chapapp{} \thechapter~\Large #1
        \vskip 1.0cm}}
    \renewcommand{\chapter}{
        \global\@topnum=0
        \@afterindentfalse
        \secdef\@chapter\@schapter}
    \renewcommand{\chaptermark}[1]{\markboth{}{\hbox to\textwidth{\it Глава \thechapter \hfil \hspace{0.1cm} #1}}}
\makeatother

\titleclass{\part}{top} % make part like a chapter
\titleformat{\part}
[display]
{\centering\normalfont\Huge\bfseries}
{\titlerule[5pt]\vspace{3pt}\titlerule[2pt]\vspace{3pt}\MakeUppercase{\partname} \thepart}
{0pt}
{\titlerule[2pt]\vspace{1pc}\huge\MakeUppercase}
%
\titlespacing*{\part}{0pt}{0pt}{20pt}

\setmainfont[Mapping=tex-text]{Times New Roman}
\setmonofont[Scale=MatchLowercase,Mapping=tex-text]{Courier New}

\begin{document}

\lstset{ %
language=Python,
basicstyle=\linespread{1.0}\small\ttfamily, % размер и начертание шрифта для подсветки кода
numbers=none,
%showspaces=false,
%showstringspaces=false,
%showtabs=false,
frame=lines,              % рисовать рамку вокруг кода
tabsize=4,                 % размер табуляции по умолчанию равен 2 пробелам
captionpos=t,
breaklines=true,
breakatwhitespace=false,
escapeinside={@}{@},
deletekeywords={sizeof},
otherkeywords={LL, vector}
}

\renewcommand{\indexname}{Алфавитный указатель}

\pagestyle{empty}
\begin{center}
$\ $
\\
\vspace{10cm}
$\ $
\\
{\Large\bf PARSER GENERATOR}
$\ $
\\
\vspace{12cm}
$\ $
\\
CSC, 2017
\end{center}
\clearpage


\pagestyle{plain}
\tableofcontents
\clearpage
\chapter{Интерфейсы}


\Section{Tree}

Интерфейс деревьев, для парсинга которых генерируется парсер. Задаёт три абстрактных метода:
\\

\begin{lstlisting}
def get_value(self, tree_path: TreePath) -> str
\end{lstlisting}
Метод возвращает значение элемента дерева, на который указывает $tree\_path$.
\\

\begin{lstlisting}
def get_elements(self, tree_path: TreePath) -> List[Tree]
\end{lstlisting}
Метод возвращает список поддеревьев, которые удовлетворяют пути $tree\_path$.
\\

\begin{lstlisting}
def get_iter(self)
\end{lstlisting}
Переопределение базового метода. Возвращает итератор по всем элементам дерева.

\Section{TreePath}

Интерфейс путей в дереве. Задаёт основные операции, которые можно производить с путями:
\\

\begin{lstlisting}
def get_relative_path(self, tree_path) -> TreePath
\end{lstlisting}

Метод возвращает путь относительно пути $tree\_path$. Необходима поддержка лишь случая, когда $tree\_path$ является префиксом текущего пути. Без обобщения на пути вида <<../../>>.
\\

\begin{lstlisting}
def get_common_prefix(self, tree_path, in_block=False) -> TreePath
\end{lstlisting}

Метод возвращает наибольший общий префикс путей. Если флак $in\_block$ выставлен, то для путей производится их обощение по индексам. Иначе определяется точное совпадение.
\\

\begin{lstlisting}
def len(self) -> int
\end{lstlisting}

Метод возвращает длину пути -- целое число.
\\

\begin{lstlisting}
def drop_for_len(self, len) -> TreePath
\end{lstlisting}

Обрезание пути до заданной длины.
\\

\begin{lstlisting}
def concat(self, tree_path) -> TreePath
\end{lstlisting}

Конкатенация путей. Входной параметр должен содержать относительный путь.
\\

\Section{Markup}

Интерфейс разметки. Необходимо определить три поля и один метод:
\\

\begin{lstlisting}
file
\end{lstlisting}

Относительный путь до файла, которому соответствует разметка.
\\

\begin{lstlisting}
type
\end{lstlisting}

Тип разметки, поддерживаемый классом $MarkupTypeRegistry$.
\\

\begin{lstlisting}
components
\end{lstlisting}

Список элементов разметки, определённых в заданном файле.
\\

\begin{lstlisting}
def add(self, component)
\end{lstlisting}

Метод добавления новой компоненты в список $components$.
\\


\Section{Algorithm}

Интерфейс основного алгоритма. Задаёт два метода, которые определяют схему взаимодействия с ним:
\\

\begin{lstlisting}
def learn(self, markup_list: List[Markup])
\end{lstlisting}

Метод обучения парсера на списке разметок. На вход принимает список разметок одного типа.
\\

\begin{lstlisting}
def parse(self, raw_page: str) -> ParserResult:
\end{lstlisting}

Метод для парсинга страницы. Следует вызывать после вызова обучения парсера. Возвращает элемент типа $ParserResult$.

\clearpage
\chapter{Реализации}




\end{document}
